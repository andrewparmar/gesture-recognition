\documentclass[../report.tex]{subfiles}
\begin{document}
    
    \begin{frame}[t]
        \frametitle{6a. Discussion}
        \begin{normalsize}
            \begin{itemize}
                \setlength\itemsep{1em}\fontsize{6pt}{6pt}
                
                \item[]{\textbf{Between all color channels, which channel, in your opinion, most resembles a gray-scale conversion of the original.  Why do you think this?  Does it matter for each respective image? (For this problem, you will have to read a bit on how the eye works/cameras to discover which channel is more prevalent and widely used)} }
                
                \item[]{\selectfont\textcolor{blue}{Splitting the southafricaflagface.png image into different channels, it appears that the green channel resembles a grayscale image the closest. From reading more about this, there seem to be two primary reasons for this:\\

1. The human visual system puts more emphasis on luminance than color. The eye can perceive more detail in the form of luminosity changes than with color changes. The human eye also has higher sensitivity to green light due to the ability of green light to stimulate two of the three types of cones in the eye. Green is, therefore, a desirable color channel to emphasize luminosity with. A characteristic that is taken advantage of by systems that are converting images to greyscale, where higher weight (~70\%) is given to the green channel.

2. Camera technology has shaped up to align with the human visual systems preferences, in particular, it is common for digital camera sensors to use the Bayer pattern which allows for two green pixels for each pair of red and blue pixels. This results in images having less noise in the green channel as compared to the red and blue channels.\\

I tried this split-channel technique on other images as well and saw a similar effect. The green channel tends to retain the most detail from the full spectrum image.}}
                
            \end{itemize}
        \end{normalsize}
    \end{frame}

    \begin{frame}[t]
        \frametitle{6b. Discussion}
        \begin{normalsize}
            \begin{itemize}
                \setlength\itemsep{1em}\fontsize{6pt}{6pt}
                
                \item[]\textbf{What does it mean when an image has negative pixel values stored?  Why is it important to maintain negative pixel values?}
                
                \item[]{\selectfont\textcolor{blue}{Negative values in an image means that the data type the image is represented by allows negative values, and that 0 in this frame of reference is not black, but rather an inbetween value. Cutting off these negative values can result in loss of detail of the image. If required, these should be normalized so that they are converted to a more standardized range, like [0, 255]}}
                
            \end{itemize}
        \end{normalsize}
    \end{frame}

    \begin{frame}[t]
        \frametitle{6c. Discussion}
        \begin{normalsize}
            \begin{itemize}
                \setlength\itemsep{1em}\fontsize{6pt}{6pt}
                
                \item[]\textbf{In question 5, noise was added to the green channel and also to the blue channel. Which looks better to you? Why? What sigma was used to detect any discernible difference?}
                
                \item[]{\selectfont\textcolor{blue}{After adding noise to the green and blue channels, the image with noise on the blue channel looks better. The colors seem to be closer to the original than the green channel image. I believe this is because the eye can spot more pixels in the green spectrum of lift and therefore perceives more noise than in the blue channel's case. This has the effect of making the image appear more grainy, and different from the original. I could start seeing a noticeable difference in quality of the images at around 7 sigma.}}
                
            \end{itemize}
        \end{normalsize}
    \end{frame}
    
\end{document}