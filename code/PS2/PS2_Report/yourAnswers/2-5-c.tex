Color masking is one of the primary stages of my detection pipeline. With the simulated images, I could dial in a color code and filter only for a desired color space. With real images, colors are no longer fixed within a tight tolerance spec. A stop signs color can be different shades of red, depending on lighting conditions, and the weathering effects on the sign. Due to this, I had to increase my tolerance ranges by a significant margin to allow for filtering the required signs. This has the negative effect of also retaining aspects of the pictures that are not the traffic sign we are interested in detecting.\\

After the color filter, my primary approach was to use cv2.HoughLinesP to get as many lines as possible in the image. I used the min and max x, y values to get the corners of the bounding box which led to determining the centroid. This method worked for simulations where the colors were controlled and localized. Within the real images this technique caused the centroid detection to be off if colors similar to the sign were present in other areas of the picture.\hfill
An area of improvement would be to use the sign angle information to restrict the line space, while also taking into consideration rotation of the sign.

\vspace{3mm} %5mm vertical space

Image sources:\\
img-5-a-1.png - Bob Kerner \url{https://bit.ly/2uX9JAb} \\
img-5-a-2.png - New Jersey Hills Media Groyp \url{https://bit.ly/31djPco}\\
img-5-a-3.png - Niagara Now \url{https://bit.ly/31e9EnX}\\

img-5-b-1.png - Shutterstock \url{https://shutr.bz/2vAGzHr}\\
img-5-b-2.png - Bob Kerner \url{https://bit.ly/2uX9JAb} \\
img-5-b-3.png - Alamy \url{https://bit.ly/2SbBywP}\\