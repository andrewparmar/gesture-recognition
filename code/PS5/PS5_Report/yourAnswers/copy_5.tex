% A line starting with a % character is a comment and wont be included in the report.
% Always use two backslashes like \\ to insert a new line. Take a look at the example below.

I ended up using the Appearance-Particle-Filter for this problem. The tick was to use a template that had enough unique features with the least amount of change from frame to frame. The torso and face sections worked quite well for this. Including the legs resulted in the template changing too much from frame to frame which caused the particle filter to jump around a lot. I used a low alpha value to take into consideration the effect of the changing background, while keeping the main features of the person we were tracking relatively consistent. I used a larger sigma value for the dynamics as the targets were moving pretty quickly. This also helped during occlusions, as the particles would disperse, but the large dynamics helped in regaining the target pretty quickly, so I didn't need to do anything special to deal with occlusions. \\
I cloned and modified run_particle_filter into run_particle_filter_multi_target that could take a list of targets. This was needed to initialize and run one particle filter for each of the targets. \\

I also tried the KalmanFilter but ran into issues with the underlying cv2.matchTemplate in run_kalman_filter that provided measurement values. Investigating what measurements were being returned showed large errors as match template returned coordinates for the wrong person in the frame. I'm sure with more tuning KalmanFilter would also have given reasonable results, specially since we were dealing with tracking constant velocity targets - however, particle filtering seemed more appealing due to the shorter implementation effort. \\
