\documentclass[../report.tex]{subfiles}
\begin{document}

    \begin{frame}
        \frametitle{3a: Haar Features}
        \begin{figure}[!htb]
            \centering
            \frame{\includegraphics[keepaspectratio,height=0.65\textheight,width=0.45\textwidth]{ps6-3-a-1}}
            \caption{ps6-3-a-1}
        \end{figure}
    \end{frame}

    \begin{frame}
        \frametitle{3a: Haar Features}
        \begin{figure}[!htb]
            \centering
            \frame{\includegraphics[keepaspectratio,height=0.65\textheight,width=0.45\textwidth]{ps6-3-a-2}}
            \caption{ps6-3-a-2}
        \end{figure}
    \end{frame}

    \begin{frame}
        \frametitle{3a: Haar Features}
        \begin{figure}[!htb]
            \centering
            \frame{\includegraphics[keepaspectratio,height=0.65\textheight,width=0.45\textwidth]{ps6-3-a-3}}
            \caption{ps6-3-a-3}
        \end{figure}
    \end{frame}

    \begin{frame}
        \frametitle{3a: Haar Features}
        \begin{figure}[!htb]
            \centering
            \frame{\includegraphics[keepaspectratio,height=0.65\textheight,width=0.45\textwidth]{ps6-3-a-4}}
            \caption{ps6-3-a-4}
        \end{figure}
    \end{frame}

    \begin{frame}
        \frametitle{3a: Haar Features}
        \begin{figure}[!htb]
            \centering
            \frame{\includegraphics[keepaspectratio,height=0.65\textheight,width=0.45\textwidth]{ps6-3-a-5}}
            \caption{ps6-3-a-5}
        \end{figure}
    \end{frame}

    \begin{frame}[t]
        \frametitle{3c: Analysis}
        \begin{normalsize}
            \begin{itemize}
                \setlength\itemsep{1em}\fontsize{6pt}{6pt}

                \item[]{\textbf{\selectfont\textcolor{blue}{ How does working with integral images help with computation time? Give some examples comparing this method and np.sum. }}}
                
                \item[]\textbf{% A line starting with a % character is a comment and wont be included in the report.
% Always use two backslashes like \\ to insert a new line. Take a look at the example below.

Integral images allow us to keep computation of areas in constant time regardless of the size of the image. The np.sum() computation has to compute sums over the complete sub-matrix while the integral image method has to just retrieve the same number of elements regardless of the matrix size.
There is an upfront cost of computing the integral image which will be proportional to image size. However this is minimal in a situation where thousands of summation calculations need to be performed. These examples show the difference in computation times using the two methods and matrices of different sizes.

HaarFeature type (2,2)

Matrix size = (10000, 10000) Summation size = (8000, 8000) np.sum: 0.0636899471, integral\_image: 0.0000143051\\
Matrix size = (20000, 20000) Summation size = (18000, 18000) np.sum: 0.7915508747, integral\_image: 0.0102751255\\
Matrix size = (40000, 40000) Summation size = (28000, 28000) np.sum: 15.1851959229, integral\_image: 0.0048100948\\}
            \end{itemize}
        \end{normalsize}
    \end{frame}
    
\end{document}