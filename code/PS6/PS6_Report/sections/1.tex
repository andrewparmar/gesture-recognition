\documentclass[../report.tex]{subfiles}
\begin{document}
    \begin{frame}
        \frametitle{1a: Average face}
        \begin{figure}[!htb]
            \centering
            \frame{\includegraphics[keepaspectratio,height=0.65\textheight,width=0.45\textwidth]{ps6-1-a-1}}
            \caption{ps6-1-a-1} 
        \end{figure}
    \end{frame}

    \begin{frame}
        \frametitle{1b: Eigenvectors}
        \begin{figure}[!htb]
            \centering
            \frame{\includegraphics[keepaspectratio,height=0.65\textheight,width=0.45\textwidth]{ps6-1-b-1}}
            \caption{ps6-1-b-1} 
        \end{figure}
    \end{frame}

    \begin{frame}[t]
        \frametitle{1c: Analysis}
        \begin{normalsize}
            \begin{itemize}
                \setlength\itemsep{1em}\fontsize{6pt}{6pt}

                \item[]{\textbf{\selectfont\textcolor{blue}{ Analyze the accuracy results over multiple iterations. Do these “predictions” perform better than randomly selecting a label between 1 and 15? Are there any changes in accuracy if you try low values of k? How about high values? Does this algorithm improve changing the split percentage p? }}}
                
                \item[]\textbf{% A line starting with a % character is a comment and wont be included in the report.
% Always use two backslashes like \\ to insert a new line. Take a look at the example below.

A random selection should have a mean accuracy of 50\%. The predictions with the naive classifier definitely perform better. The average performance over six runs using a split p=0.8 and k=7 is ~64\%. Using k values lower than 5 showed a reduction in accuracy. The accuracy plateaus around k=7.\\
The p value doesn't appear to be as sensitive to the accuracy. Using a value of p=0.4 still provided an average accuracy of 64\%, and accuracy values of ~50\% for p as low as 0.2. Raising p to greater than 0.5 showed no significant improvements in accuracy.

}
            \end{itemize}
        \end{normalsize}
    \end{frame}
    
\end{document}