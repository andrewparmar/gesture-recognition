LK seems to start performing poorly for large displacements, and seems to get progressively worse the larger the displacement gets. This is expected as the Taylor series expansion for the solution to the brightness consistency constraint no longer holds for large displacement values. As only the first order Taylor expansion is used, the  truncation errors will be now be large. My Lukas-Kanade optic flow implementation already included a derivative blurring step. This blur used a gaussian kernel. I had to double the sigmaX value for this kernel (double compared to part 1a) to generate a smoother quiver plot.